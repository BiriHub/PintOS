\documentclass[11pt]{scrartcl}
\newcommand*\student[1]{\newcommand{\thestudent}{{#1}}}
\newcommand*\course[1]{\newcommand{\thecourse}{{#1}}}
\newcommand*\courseshort[1]{\newcommand{\thecourseshort}{{#1}}}
\newcommand*\assnumber[1]{\newcommand{\theassnumber}{{#1}}}
\setcounter{secnumdepth}{4}

\student{}
\course{}
\courseshort{} 
\assnumber{}

%----------------------------------------------------------------------------------------
%	PACKAGES AND OTHER DOCUMENT CONFIGURATIONS
%----------------------------------------------------------------------------------------

\usepackage[utf8]{inputenc} % Required for inputting international characters
\usepackage[T1]{fontenc} % Use 8-bit encoding
\usepackage[sc]{mathpazo}
\usepackage{caption, subcaption}
\usepackage[hidelinks]{hyperref}
\usepackage{inconsolata}
\usepackage[svgnames,table]{xcolor}
\usepackage{booktabs}
\usepackage{boldline}
% \usepackage{minted}
\usepackage{mathtools}
\usepackage{chemfig}
% \usepackage{minted}
\usepackage{caption}
% \usepackage{subcaption}

	
\usepackage{color, colortbl}
\definecolor{LightCyan}{rgb}{0.88,1,1}
\definecolor{LightGreen}{rgb}{1,0.88,1}
\definecolor{LightRed}{rgb}{0.8,1,0.90}
\definecolor{LightBoh}{rgb}{0.3,0.7,0.88}


\usepackage[english]{babel} % English language hyphenation
\usepackage{amssymb}
\usepackage{amsmath}% Math packages
\usepackage{bm}
\usepackage{listings} % Code listings, with syntax highlighting
\usepackage{graphicx} % Required for inserting images
\usepackage{float}

% \usepackage{minted}

\usepackage{courier} %% Sets font for listing as Courier.
\usepackage{listings, xcolor}

\lstset{
tabsize = 4, %% set tab space width
showstringspaces = false, %% prevent space marking in strings, string is defined as the text that is generally printed directly to the console
numbers = left, %% display line numbers on the left
commentstyle = \color{green}, %% set comment color
keywordstyle = \color{blue}, %% set keyword color
stringstyle = \color{red}, %% set string color
rulecolor = \color{black}, %% set frame color to avoid being affected by text color
basicstyle = \small \ttfamily, %% set listing font and size
breaklines = true, %% enable line breaking
numberstyle = \tiny,
}


%             \begin{lstlisting}[language = Java, frame = single, firstnumber = last, escapeinside={(*@}{@*)}]
% % 
%             \end{lstlisting}
%----------------------------------------------------------------------------------------
%	DOCUMENT MARGINS
%----------------------------------------------------------------------------------------


\usepackage{geometry} % For page dimensions and margins
\geometry{
	paper=a4paper, 
	top=2.5cm, % Top margin
	bottom=3cm, % Bottom margin
	left=2cm, % Left margin
	right=2cm, % Right margin
}
\setlength\parindent{0pt}

%----------------------------------------------------------------------------------------
%	SECTION TITLES
%----------------------------------------------------------------------------------------

\usepackage{sectsty}
\sectionfont{\vspace{6pt}\centering\normalfont\LARGE\scshape}
\subsectionfont{\normalfont\Large\bfseries} % \subsection{} styling
\subsubsectionfont{\normalfont\large\itshape} % \subsubsection{} styling
\paragraphfont{\normalfont\scshape} % \paragraph{} styling

%----------------------------------------------------------------------------------------
%	HEADERS AND FOOTERS
%----------------------------------------------------------------------------------------

\usepackage{scrlayer-scrpage}
\ofoot*{\pagemark} % Right footer
\ifoot*{} % Left footer
\cfoot*{\thecourseshort \  \theassnumber} % Centre footer

%----------------------------------------------------------------------------------------
%	TITLE SECTION
%----------------------------------------------------------------------------------------

\title{	
    \vspace{-3cm} 
    \begin{figure}[H]
        \hspace{-2cm}
    	\includegraphics[width=65mm]{template/usi-inf-logo.png}
    \end{figure}
    \vspace{1cm} 
	\normalfont\large
	\textsc{Operating Systems\thecourse}\\
	\vspace{5pt}
	\rule{\linewidth}{0.5pt}\\
	\vspace{10pt}
	{\LARGE PintOS: Report Project \textbf{3} \theassnumber}\\
	\vspace{5pt}
	\rule{\linewidth}{0.5pt}\\
	\vspace{5pt}
}

\author{
	\LARGE
	Group 12 \\
	Michelangelo Bettini, Leonardo Birindelli, Alessandro della Flora
}

\date{\normalsize Spring Semester, \today}

\begin{document}


\maketitle

\begin{table}[h]
\begin{center}
\scalebox{0.8} {%
\begin{tabular}{|p{0.02cm}p{16cm}|}
\hline
&\\
\multicolumn{2}{|c|}{\Large\textbf{ Report Instructions}}\\
&\\
\textbullet & Please report \textbf{all} changes, even if minor, that you did to complete the project.\\
\textbullet & You have to list all files that have been modified, and for each of them, list all functions/structs that have been modified or added (clearly stating "modified"/"added"). Then add a brief explanation or motivation for all those  changes.\\
\textbullet & A single report is required for each group. For the first individual project, each student submits a report together with the source code files that were changed.\\
\hline
\end{tabular}
}
\end{center}
\end{table}

\section{Files changed }
\begin{itemize}
    \item pintos/threads/thread.h
    \item pintos/threads/thread.c 
    \item pintos/userprog/syscall.c
    \item pintos/userprog/process.c
\end{itemize}

\section{Changes}

\subsection*{pintos/threads/thread.h}

\begin{itemize}
	\item struct thread (\textit{implemented}):\newline
    It has been added to the struct three new fields in the struct section USERPROG owned by the userprog/process.c : \texttt{struct thread *parent} (ln. 108) represents the reference to the parent thread of the actual thread; \texttt{bool is\_parent\_waiting} (ln. 109) checks if the parent is waiting for this thread to exit or not; 
		\texttt{int exit\_status} (ln. 110) represents the exit status of the thread.
	\item struct thread *thread\_get\_by\_tid (int tid) (\textit{added}):\newline
	(ln. 168) auxiliary function used to obtain the thread specified with the given identifier.
    
\end{itemize}

\subsection*{pintos/threads/thread.c}

\begin{itemize}
	\item struct thread *thread\_get\_by\_tid (int tid) (\textit{implemented}):\newline
	(ln. 570) auxiliary function used to obtain the thread specified with the given identifier.
\end{itemize}

\subsection*{pintos/userprog/syscall.c}

\begin{itemize}
	\item static void syscall\_handler (struct intr\_frame *f) (\textit{implemented}):\newline
	(ln. 23) implements the syscall handler function so as it is able to handle the \textit{exit} and \textit{write} system calls.
\end{itemize}

\subsection*{pintos/userprog/process.c}

\begin{itemize}
	%TODO
	\item static void start\_process (void *func) (\textit{implemented}):\newline
	\item static void start\_process (void *func) (\textit{implemented}):\newline
	\item int process\_wait (tid\_t child\_tid) (\textit{implemented}):\newline
	\item void process\_exit (void) (\textit{implemented}):\newline
	\item void push\_stack (void ** top\_of\_stack, void * p\_data, int len) (\textit{added}):\newline
	\item void	push\_args (void** top\_of\_stack, char * cmd) (\textit{added}):\newline
	
	% \item static void syscall\_handler (struct intr\_frame *f) (\textit{implemented}):\newline
	% (ln. 23) implements the syscall handler function so as it is able to handle the \textit{exit} and \textit{write} system calls.
\end{itemize}


\end{document}
